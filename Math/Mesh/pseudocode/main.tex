\documentclass[letter,10pt]{article}
\usepackage{algorithm} 
\usepackage{algpseudocode} 
\begin{document} 

Variables de entrada:
\begin{itemize}
	\item  $\alpha 1$ - Ángulo de la primera línea
	\item  $\alpha 2$ - Ángulo de la segunda línea
	\item  tamaño - Tamaño de la figura
	\item  puntos - Número de puntos en cada línea
	\item  origen - Punto de origen
\end{itemize}

\begin{algorithm}
	\caption{Figura 1} 
	\begin{algorithmic}[1]
		\State stepsize = tamaño/puntos; // definir separación entre puntos
		\State $\theta 1$ = $\alpha 1*(\pi/180);$ // convertir $\alpha 1$ a radianes
		\State $\theta 2$ = $\alpha 2*(\pi/180);$ // convertir $\alpha 2$ a radianes
		\For {$(i = 0; i < puntos+1; i = i+1)$} // para cada punto
			\State steps = $i*step;$ // número de pasos
			\State // Del primer al último punto calcular el punto correspondiente respecto al ángulo
			\State $punto1.X= origen.X+cos(\theta 1)*steps; $
			\State $punto1.Y= origen.Y-sin(\theta 1)*steps; $
			\State // Del último al primer punto calcular el punto correspondiente respecto al ángulo
			\State $punto2.X= origen.X+cos(\theta 2)*(tamano-steps); $
			\State $punto2.Y= origen.Y-sin(\theta 2)*(tamano-steps); $
			\State Dibujar línea del punto1 al punto2
		\EndFor
	\end{algorithmic} 
\end{algorithm}
\end{document}