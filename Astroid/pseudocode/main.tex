\documentclass[letter,10pt]{article}
\usepackage{algorithm} 
\usepackage{algpseudocode} 
\begin{document} 

Variables de entrada:
\begin{itemize}
	\item  escalera = Trazo inicial
	\item  tamaño - Tamaño de las línas perpendiculares
	\item  puntos - Número de puntos
	\item tamañoelipse - Tamaño de la elipse
\end{itemize}

\begin{algorithm}
	\caption{Figura 2} 
	\begin{algorithmic}[1]

		\State Trazar líneas perpendiculares
		\State H = hipotenusa(escalera); // Calcular la hipotenusa del triángulo formado por la escalera (Magnitud inicial)
		\State stepsize = tamaño/points; // Separación entre líneas

		\For {$(i = 0; i < puntos+1; i = i+1)$} // Para cada punto

			\State // Calcular nuevo triángulo en Y
			\State  C1 = (stepsize*i);   // Actualizar tamaño de Cateto 1 (En Y)
			\State  C2 = pitágoras(C1,H); // Calcular Cateto 2 dado C1 y H
			\State  Reflejar triángulo
			\State 	Trazar Hipotenusas de los nuevos dos tríangulos
			
			\If {(dibujarElipse)} //Decidir si dibujar Elipse
				\State Calcular un punto sobre las Hipotenusas anteriores a una distancia de tamañoelipse desde uno de los estremos de las Hipotenusas
				\State Dibujar puntos anteriores
			\EndIf

			\State // Calcular nuevo triángulo en X
			\State  C1 = (stepsize*i);   // Actualizar tamaño de Cateto 1 (En X)
			\State  C2 = pitágoras(C1,H); // Calcular Cateto 2 dado C1 y H
			\State  Reflejar triángulo
			\State 	Trazar Hipotenusas de los nuevos dos tríangulos
			
			\If {(dibujarElipse)} //Decidir si dibujar Elipse
				\State Calcular un punto sobre las Hipotenusas anteriores a una distancia de tamañoelipse desde uno de los estremos de las Hipotenusas
				\State Dibujar puntos anteriores
			\EndIf
		\EndFor
	\end{algorithmic} 
\end{algorithm}
\end{document}